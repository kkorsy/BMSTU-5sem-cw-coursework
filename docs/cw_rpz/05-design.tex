\chapter{Конструкторский раздел}

\section{Общий алгоритм решения поставленной задачи}

Структура программы представлена на рисунке \ref{img:idf0} в виде IDF0-диаграммы.

\includeimage
{idf0} 
{f}
{H}
{0.9\textwidth} 
{Структура программы}

В соответствии с рисунком~\ref{img:idf0} входными данными для разрабатываемой программы будут являться данные об объектах, об источнике света, о расположении камеры. В программе будут реализованы следующие алгоритмы: модифицированный алгоритм, использующий z-буфер, алгоритм закраски Гуро. В результате работы программы будет получен кадр сцены, содержащий модель цветка, а также время  создания одного кадра.

\section{Алгоритм, использующий z-буфер}

Схема алгоритма, использующего z-буфер, представлена на рисунке \ref{img:zBuf}

\includeimage
{zBuf} 
{f}
{H}
{0.6\textwidth} 
{Схема алгоритма, использующего z-буфер}

\section{Модифицированный алгоритм, использующий z-буфер}

Схема модифицированного алгоритма, использующего z-буфер, представлена на рисунках \ref{img:zBufModif1}--\ref{img:zBufModif2}

\includeimage
{zBufModif1} 
{f}
{H}
{0.33\textwidth} 
{Схема модифицированного алгоритма, использующего z-буфер (часть 1)}

\includeimage
{zBufModif2} 
{f}
{H}
{0.9\textwidth} 
{Схема модифицированного алгоритма, использующего z-буфер (часть 2)}

\section{Алгоритм закраски Гуро}

В алгоритме закраски Гуро сначала определяется интенсивность в вершинах, потом вдоль ребер вычисляется интенсивность соответствующего пикселя. Схема алгоритма представлена на рисунке \ref{img:guro}.

\includeimage
{guro} 
{f}
{H}
{0.4\textwidth} 
{Схема алгоритма закраски Гуро}

Этот метод хорошо сочетается с алгоритмом, использующим z-буфер. Для каждой сканирующей строки определяются ее точки пересечения с ребрами. В этих точках интенсивность вычисляется с помощью линейной интерполяции интенсивностей в вершинах ребра. Затем для всех пикселей, находящихся внутри многоугольника и лежащих на сканирующей строке, аналогично вычисляется интенсивность.

\section{Схема алгоритма генерации одного кадра изображения}

На рисунках \ref{img:fullAlg2}--\ref{img:fullAlg1} представлена схема алгоритма генерации одного кадра изображения, объединяющего в себе модифицированный алгоритм Z-буфера и алгоритм закраски по Гуро.

\includeimage
{fullAlg2} 
{f}
{H}
{0.25\textwidth} 
{Схема алгоритма генерации одного кадра изображения (часть~1)}

\includeimage
{fullAlg1} 
{f}
{H}
{0.9\textwidth} 
{Схема алгоритма генерации одного кадра изображения (часть~2)}

\section{Описание используемых структур данных}

На рисунке~\ref{img:classes} представлены классы основных объектов сцены, и показаны их атрибуты.

\includeimage
{classes} 
{f}
{H}
{0.9\textwidth} 
{Классы основных объектов сцены}

Основные использованные типы и структуры данных:
\begin{itemize}[label=---]
	\item тип \textit{double} для координат точки, вектора, а также для интенсивности источника;
	\item класс \textit{Вектор} для вектора нормали, направления камеры;
	\item массив объектов класса \textit{Вершина} для использования в классах \textit{Сторона} и \textit{Модель};
	\item класс \textit{Точка} для положения камеры, источника света, а также для центра модели;
	\item массив объектов класса \textit{Сторона} для использования в классе \textit{Модель};
	\item тип \textit{Qrgb} для цвета.
\end{itemize}

\section*{Вывод}

В данном разделе был представлен общий алгоритм решения поставленной задачи в виде диаграммы IDF0 0 уровня, схемы алгоритмов использующего z-буфер, его модификации для построения теней и закраски Гуро. Также был описан алгоритм генерации одного кадра изображения.