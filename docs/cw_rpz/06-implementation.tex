\chapter{Технологический раздел}

\section{Средства реализации}

В качестве языка для разработки программы был выбран язык программирования C++. Данный выбор основан на следующих аспектах:

\begin{itemize}[label=---]
	\item В стандартной библиотеке языка присутствует поддержка всех структур данных, выбранных по результатам проектирования;
	\item Средствами языка можно реализовать все алгоритмы, выбранные в результате проектирования;
	\item C++ обладает высокой вычислительной производительностью, что очень важно для выполнения поставленной задачи;
	\item Статическая типизация позволит устранять ошибки на стадии компиляции;
	\item Доступность учебной литературы.
\end{itemize}

В качестве среды разработки был выбран QtCreator. Данный выбор обусловлен следующими факторами:

\begin{itemize}[label=---]
	\item Данная среда разработки предоставляет удобную графическую библиотеку;
	\item Позволяет работать с графическим интерфейсом;
	\item Является бесплатной.
\end{itemize}

\section{Разработка используемых классов}

На рисунке~\ref{img:uml} представлена схема взаимодействия основных объектов сцены и показаны их составляющие.

Используются следующие основные классы:

\begin{itemize}[label=---]
	\item Point --- класс точки в трехмерном пространстве;
	\item Vector --- класс вектора в трехмерном пространстве;
	\item Vertex --- класс вершины;
	\item Side --- класс грани;
	\item Model --- класс модели;
	\item Camera --- класс камеры;
	\item LightSource --- класс источника света;
	\item Receptacle --- класс цветоложа цветка;
	\item Leaf --- класс листа цветка;
	\item Petal --- класс лепестка цветка;
	\item Stem --- класс стебля цветка;
	\item Center --- класс центральной части цветка;
	\item Surface --- класс ограничивающей поверхности.
\end{itemize}

\section{Разработка интерфейса}

В связи с тем, что у пользователя должна быть возможность перемещать камеру и источник света, в интерфейсе необходимы кнопки, при нажатии на которые будет происходить соответствующее движение. Для камеры необходимы кнопки для перемещения по всем трем осям, а также кнопки для вращения вокруг них. Для источника света достаточно только кнопок для перемещения, поскольку в данном случае сцена не изменится от его поворота.

Для перемещения вдоль оси $x$ в положительном и отрицательном направлениях предусмотрены кнопки <<Вправо>> и <<Влево>> соответственно, вдоль $y$ --- <<Вверх>>, <<Вниз>>, вдоль $z$ --- <<Ближе>>, <<Дальше>>. Для поворота вокруг оси $x$ на положительный и отрицательный угол предусмотрены кнопки <<Вниз>> и <<Вверх>> соответственно, вокруг $y$ --- <<Вправо>>, <<Влево>>.

В результате пользователю предоставляется интерфейс, показанный на рисунке~\ref{img:interface}. На панели справа расположены кнопки для перемещения и вращения камеры, перемещения источника света. В нижней части находятся кнопки, позволяющие сменить время суток: день, ночь.

\includeimage
{interface} 
{f}
{H}
{0.9\textwidth} 
{Интерфейс программного обеспечения} 

При нажатии на кнопку <<Ночь>> лепестки модели цветка начнут вращаться к центру, имитируя закрытие цветка, начнет затемняться фон, а также снижаться интенсивность источника. При нажатии на кнопку <<День>> будут происходить обратные действия: раскрытие лепестков, фон будет становиться светлее, увеличение интенсивности источника. Нажатие на кнопки <<День>> и <<Ночь>> возможно, если модель цветка не находится в процессе смены времени суток.

В верхнем левом углу расположено выпадающее меню, в котором находятся кнопки <<Горячие клавиши>> и <<О программе>>. При нажатии на кнопку <<Горячие клавиши>> пользователю будет предоставлена таблица, связывающая клавиши клавиатуры и кнопки интерфейса~(Рисунок~\ref{img:hotkeys}).

\includeimage
{hotkeys} 
{f}
{H}
{0.9\textwidth} 
{Горячие клавиши программного обеспечения}

\section*{Вывод}

В данном разделе было разработано программное обеспечение для визуализации модели цветка. Был выбран язык программирования и среда разработки, а также описан интерфейс, предоставляемый пользователю.
