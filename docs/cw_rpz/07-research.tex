\chapter{Исследовательский раздел}

В связи с тем, что все объекты сцены задаются с использованием полигональной сетки, возникает вопрос о том, как зависит время генерации одного кадра изображения от величины шага полигональной сетки.

\section{Технические характеристики}

Технические характеристики устройства, на котором выполнялось исследование представлены далее:
\begin{itemize}[label={---}]
	\item операционная система: Windows 11, x64;
	\item оперативная память: 8 Гб;
	\item процессор: AMD Ryzen 5 5500U с видеокартой Radeon Graphics 2.10~ГГц.
\end{itemize}

Во время исследования ноутбук был нагружен только встроенными приложениями окружения.

\section{Цель исследования}

Целью исследования является определение зависимости скорости генерации одного кадра изображения от шага полигональной сетки. 

Исследование проводится при смене времени суток с значения <<День>> на <<Ночь>>. В качестве результирующего значения времени генерации одного кадра изображения берется среднее значение. Шаг полигональной сетки принимает следующие значения: $\{0.025, 0.050, 0.100, 0.250, 0.500, 0.750\}$.

\section{Результаты исследования}

Результаты исследования представлены в таблице~\ref{tbl:mes}.

\begin{table}[H]
	\begin{center}
		\captionsetup{justification=raggedright, singlelinecheck=off}
		\caption{Результаты замеров времени}
		\label{tbl:mes}
		\begin{tabular}{|r|r|}
			\hline
			Шаг сетки & Время генерации одного кадра, мс\\
			\hline
			0.025 & 50 585\\
			\hline
			0.050 & 5 468 \\
			\hline
			0.100 & 1 964\\
			\hline
			0.250 & 259\\
			\hline
			0.500 & 119\\
			\hline
			0.750 & 103\\
			\hline
		\end{tabular}
	\end{center}
\end{table}

На рисунке~\ref{img:graph} приведен график зависимости времени генерации одного кадра изображения от шага полигональной сетки.

\includeimage
{graph} 
{f}
{H}
{0.9\textwidth} 
{Визуализация результатов исследования} 

На рисунках~\ref{img:minGrid}~--~\ref{img:maxGrid} представлены получаемые изображения при разном шаге полигональной сетки: 0.025, 0.250 и 0.750 соответственно.

\includeimage
{minGrid} 
{f}
{H}
{0.9\textwidth} 
{Изображение, получаемое при шаге полигональной сетки 0.025} 

\includeimage
{midGrid} 
{f}
{H}
{0.9\textwidth} 
{Изображение, получаемое при шаге полигональной сетки 0.250} 

\includeimage
{maxGrid} 
{f}
{H}
{0.9\textwidth} 
{Изображение, получаемое при шаге полигональной сетки 0.750} 


\section*{Вывод}

На основании результатов исследования можно сделать вывод, что при уменьшении шага полигональной сетки возрастает время генерации одного кадра изображения. Так например, при уменьшении шага с 0.250 до 0.100 время генерации возросло в $\approx 7.6$ раза, а при уменьшении с 0.100 до 0.050 --- в $\approx 2.8$ раза.
Сравнивая рисунки~\ref{img:minGrid}~--~\ref{img:maxGrid} можно заметить, что при увеличении шага полигональной сетки лепестки приобретают более ровную, прямую форму, центральная часть становится менее круглой, появляются грани. Кроме того, снижается плавность смены оттенков при переходе от тени к свету (наиболее заметно на центральной части цветка). Все эти факторы влияют на реалистичность изображения: при увеличении шага полигональной сетки снижается реалистичность получаемого изображения.

Таким образом, оптимальным значением шага полигональной сетки является 0.250, поскольку такой шаг обеспечивает достаточную реалистичность и время генерации одного кадра изображения.

